\documentclass[unrestricted]{meetingnotesminutes}

\title{Mathworks Minidrone Competition}
\author{Vi Kumar}
\project{Mathworks Minidrone Competition}
\wheremeeting{Online Meeting on Microsoft Teams}
\whenmeeting{May 13, 2020 14:00}

\initiator{Vi Kumar}
\participant[present]{Abdullah Sherif - as394@hw.ac.uk}
\participant[present]{Vishakh Kumar - vpk2@hw.ac.uk}
\participant[present]{Vishnu Sarathy - vks2@hw.ac.uk}
\participant[information]{Dr Mehdi Nazarinia}
\participant[information]{Dr Ityonna Amber}

\begin{document}
\frontmatter

\section*{Agenda}
\begin{itemize}
  \item Prepare \LaTeX\ class for meeting minutes
  \item Hold meeting
  \item Write minutes
  \item Compile with Xe\LaTeX\ or Lua\LaTeX
\end{itemize}

\section*{Meeting minutes}
This class allows to write meeting minutes in the MEETINGNOTES official style.
It is a subclass of \texttt{meetingnotesdoc}, so see its documentation too.

Title, author and date are set with \LaTeX's usual commands
\verb|\title|, \verb|\author| and \verb|\date|.

Several other options are self-documenting and will default to a useful
tooltip à la \texttt{set with \textbackslash command}.

The author is always first in the participants list, and is marked automatically
as present.
Other participants can be added with the \verb|\participant| command, which
takes \emph{one} of the following optional arguments: \verb|present|, \verb|absent|
and \verb|information|.
For example,
\begin{verbatim}
\participant[present]{\LaTeX\ users}
\end{verbatim}
Participants should be set in the preamble.

The task list is defined by writing the tasks with the \verb|\task| command, which
takes three arguments for task description, responsible and due date, as
follows:
\begin{verbatim}
\task{Learn \LaTeX}{Word users}{ASAP}
\end{verbatim}
To print out the task list, use the \verb|\tasklist| command.
The task list is numbered automatically.

\task{Learn \LaTeX}{Word users}{ASAP}
\task{Check this template}{\TeX perts}{Anytime}
\task{Enjoy}{\LaTeX\ users}{From now on}
\tasklist

\end{document}
